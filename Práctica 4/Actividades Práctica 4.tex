\documentclass[16]{article} %[Tamaño de página y número de fuente]. %Tipo del documento
\usepackage{amsmath} %Añade comandos para ecuaciones
\usepackage[spanish]{babel} %Indica idioma del documento
\usepackage[utf8]{inputenc} %Permite escribir caracteres particulares del español (tildes, ñ, ...). Equivalente a poner \'a
\usepackage{vmargin} %Para editar márgenes
\usepackage{graphicx} %Permite instertar imágenes
\usepackage{tikz} %Permite insertar elementos gráficos
\usepackage[linesnumbered]{whilecode2} %Para escribir código en WHILE
\usepackage{listings} %Para añadir códigos de programas

\begin{document} %Crea el entorno del documento
\begin{titlepage} %Portada
	\centering  %Centra el texto
	{\bfseries\Huge Teoría de Autómatas y Lenguajes Formales\par}
   	\vspace{2cm} %Crea un espacio vertical
    {\bfseries\huge Actividades Práctica 4\par}
    \vfill %Rellena el espacio para ocupar la página entera
    {\huge Iván Romero Molina\par}
    \vspace{1cm}
    {\Large Universidad de Málaga\par}
    \vspace{1cm}
    {\large 26 de diciembre de 2022\par}
\end{titlepage}
    
\newpage
%\thispagestyle{empty} Quita el número de página
\section*{Ejercicio 1}
\noindent
Implemente en un programa WHILE que realice la suma de tres valores, se debe usar una variable auxiliar que acumule el resultado de la suma.\\\\
Código en WHILE:
\whileprogram{Q}{0}{
\Copy*{\Var{$X_2$}}{\Var{$X_1$}}

\Incr{\Var{$X_2$}}{1}
\WhileSC{\Var{$X_2$}$\neq0$}
{
	\Copy*{\Var{$X_1$}}{1}
}
}{s}
\section*{Ejercicio 2}
\noindent
Crea un script en Octave que enumere todos los vectores:
\begin{lstlisting}
function printVectors(x)
for i=0:x-1
disp(['( num2str(godeldecoding(i)) ')'])
end
end
\end{lstlisting}
\section*{Ejercicio 3}
\noindent
Crea un script en Octave que enumere todos los programas WHILE:
\begin{lstlisting}
function printWhilePrograms(x)
for i=0:x-1
disp(N2WHILE(i))
end
end
\end{lstlisting}
\end{document}